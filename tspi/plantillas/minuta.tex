\documentclass{article}

\begin{document}
\title{\bfseries Minuta de la Reuni\'{o}n \#1}
\date{}
\maketitle

\noindent{\large\bfseries Fecha:} Martes 23 de Septiembre de 2014\\
\noindent{\large\bfseries Hora de inicio:} 3:30 PM\\
\noindent{\large\bfseries Hora de finalizaci\'{o}n:} 4:30 PM\\
\noindent{\large\bfseries Lugar:} Oficinas GBH\\[0.3in]
\noindent{\large\bfseries Objectivos}\\[0.15in] Presentar al cliente el equipo que trabajar\'{a} en el proyecto, identificar el o los problemas y
aclarar las dudas existentes por medio de preguntas.\\[0.3in]
\noindent{\large\bfseries Cliente}\\
\indent Jos\'{e} Bonetti\\[0.3in]
\noindent{\large\bfseries Equipo}\\
\indent Alan Alvarez\\
%\indent Jusn\'{e}n Volquez\\
%\indent Jose Gabriel Reyes\\
%\indent Gilven Wu\\
\indent H\'{e}ctor V\'{a}squez Rom\'{a}n
\section{Temas Tocados}
Mediante la reuni\'{o}n el cliente nos dio a conocer cuales son sus problemas utilizando la herramienta \textbf{RedMine} para manejar sus proyectos.
Entre los inconvenientes expuestos est\'{a}n:
\begin{description}
\item[Configuraci\'{o}n de los proyectos] \hfill \\
El cliente realiza la configuraci\'{o}n de los proyectos por medio de un archivo de excel en el cual debe realizar
el c\'{a}lculo de los tiempos estimados y la calendarizaci\'{o}n de las tareas de manera manual. Todo este proceso se dificulta a\'{u}n m\'{a}s debido a que
este no conoce la disponibilidad real de sus recursos.
\item[Disponibilidad de los recursos al d\'{i}a] \hfill \\
La herramienta \textbf{RedMine} no le permite al cliente establecer la cantidad de horas disponibles al d\'{i}a de cada recurso.
\item[Disponibilidad de los recursos en un proyecto] \hfill \\
La herramienta \textbf{RedMine} no le permite al cliente establecer que porcentaje de tiempo estar\'{a} un determinado recurso comprometido a un proyecto.
\end{description}
\section{Temas Pendientes}
\begin{enumerate}
\item Solicitar v\'{i}a correo electr\'{o}nico las versiones del sistema operativo y base de datos donde est\'{a} alojada la herramienta \textbf{RedMine},
as\'{i} como tambi\'{e}n, la versi\'{o}n de \textbf{RedMine} y del importador que configura el proyecto en \textbf{RedMine}.
\item Solicitar v\'{i}a correo electr\'{o}nico un ejemplo del documento que se le pasa como entrada al importador que configura el proyecto en \textbf{RedMine}.
\end{enumerate}
\end{document}
